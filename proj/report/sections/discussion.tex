\section{Discussion}

In conclusion, topological analysis shows that, in this particular multisensory dataset, there might exist low dimensional geometric topology ($d \approx 2-10$) in Purkinje cell population at spontaneous and in response to sensory stimuli, by analyzing the variation of information distance metric of a short window after stimulus onset. However, future analysis should look at distance (or similarity) metric of the window after stimulus onset but also conditioned on the baseline window before the onset, e.g. partial correlation coefficient. Additionally, future analysis should consider block models of smaller block size and other null models.

Moreover, comparison between different stimulus categories shows that, at least when looking at the mean, different stimulus modalities affects the topology in a ``mixed'' way. In this particular dataset, somatosensory inputs (P) seem to induce more cycles and somewhat longer cycles than other modalities. It is to be seen whether this effect exists only in this area of recordings and in this individual mouse. It is possible that these effects are arbitrary, but might be sufficiently different for decoding purposes.

However, attempts at decoding the stimulus combinations using a simple MLP framework reveal that perusing features from TDA would only moderately increase performance when combined with activity-based features. Better performance might be gained by better construction of activity-based features alone without the need for TDA features. Future attempts should include multi-label classification goals and other simpler models like GLM, as well as regularization options during training. Additionally, features from persistent landscapes should also be considered.

\section{Abstract}

In this project, I used topological data analysis (TDA) on Purkinje cell population activity during calcium imaging of awake mouse cerebellum under different stimulus conditions. I compared the topology of Purkinje network using variation of information as the distance metric with different null models and across different stimulus conditions. The results reveal low-dimensional geometric structures in the network and different stimulus modalities have interacting effects on the topology of the functional network under combinations of stimuli. However, using a multiperceptron (MLP) network reveals that using topology-based features do not have significant meaningful gain in decoding stimulus combinations (categories) in addition to using activity-based features.